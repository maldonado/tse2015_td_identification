\documentclass[conference]{IEEEtran}

\usepackage{amssymb,amsmath}
\usepackage{wrapfig}
\usepackage{multirow}
\usepackage{graphicx}
\usepackage{algorithm}
\usepackage{algorithmic}
\usepackage{times}
\usepackage{cite}
\usepackage{url}
\usepackage{booktabs}
\usepackage{subfigure}
\usepackage{fancybox}
\usepackage{color}
\usepackage{array}
\usepackage{subfigure}
\usepackage{balance}
\usepackage{epstopdf}
\usepackage{array}
\usepackage{xspace}


\newcommand{\emad}[1]{\textcolor{red}{{\it [Emad: #1]}}}
\newcommand{\nikos}[1]{\textcolor{red}{{\it [Nikos: #1]}}}
\newcommand{\everton}[1]{\textcolor{blue}{{\it [Everton: #1]}}}

\newcommand{\todo}[1]{\colorbox{yellow}{\textbf{[#1]}}}

\newcommand{\conclusionbox}[1]{%
	\vspace{2mm}
	\framebox[0.45\textwidth][c]{%
		\parbox[b]{0.42\textwidth}{%
			{\it #1}
		}
	}
	\vspace{2mm}
}
\newcommand{\rqi}{\textbf{RQ1. What comment patterns indicate self-admitted design technical debt? How are these comment patterns different than previously proposed comment patterns?\\}}
\newcommand{\rqii}{\textbf{RQ2. Can we effectively detect self-admitted design technical debt using the proposed comment patterns?\\}}
\newcommand{\rqiii}{\textbf{RQ3. How much of the detected self-admitted design technical debt can we automatically address with state-of-the-art refactoring techniques?\\}}

\newcommand{\SADTD}{Self-admitted Design Technical Debt\xspace}

\begin{document}
\title{Using Source Code Comments to Detect \SADTD}

\author{\IEEEauthorblockN{Everton da S. Maldonado,
Nikolaos Tsantalis and
Emad Shihab}

\IEEEauthorblockA{Department of Computer Science and Software Engineering\\Concordia University,
Montreal, Canada\\
\url{e_silvam@encs.concordia.ca}, \url{nikolaos.tsantalis@concordia.ca},
\url{emad.shihab@concordia.ca}}
}

\maketitle

\begin{abstract}
During the development and maintenance of a software system, developers face unpredictable difficulties or pressures, and in many cases are forced to apply unconventional solutions to overcome these difficulties. For example, they might adopt insufficiently tested or temporary solutions (i.e., workarounds and hacks), neglect good design practices, and introduce inaccurate or incomplete documentation
due to time constraints and pressure to meet deadlines. This phenomenon has been explained through the metaphor of Technical Debt.

Prior work has shown that one of the most impacting types of technical debt is design debt and that code comments embedded in the code can be used to detect \emph{self-admitted} technical debt. Therefore, in this paper our main goal is to study \SADTD.
More specifically, we derive comment patterns that can be used to detect \SADTD. Then, we perform a case study to determine the effectiveness of our approach at detecting \SADTD. We also compare the effectiveness of our approach to prior approaches that use code smells to detect design technical debt and quantify how much of the self-admitted design debt can be automatically refactored with refactoring tools. We suggest 176 different comment patterns that can be used to detect \SADTD. Our approach can achieve precision and recall values between 74.07-96.30\% and 10.87-83.87\%, respectively. We also show that our approach detects design technical debt that is different from alternative state-of-the-art techniques used for finding design technical debt. Lastly, our findings also show that 24.58\% of the \SADTD is detected in the form of refactoring opportunities by a state-of-the-art refactoring recommendation tool. 
\end{abstract}

\IEEEpeerreviewmaketitle

\section{Introduction}
\label{sec:introduction}
Developers often have to deal with conflicting goals that require software to be delivered quickly, with high quality, and on budget. In practice, achieving all of these goals at the same time can be challenging, causing a tradeoff to be made. Often, these tradeoffs lead developers to take \emph{shortcuts} or use \emph{workarounds}. Although such shortcuts help developers in meeting their short-term goals, they may have a negative impact in the long-term.

Technical debt is a metaphor that has been used to express sub-optimal solutions that are taken consciously in a software project in order to achieve some short-term goals. Generally, these decisions allow the project to move faster in the short-term, but introduce an increased cost (i.e., debt) to maintain this software in the long run~\cite{Seaman2011,Kruchten2013IWMTD}. Prior work showed that technical debt is widespread in the software domain, is unavoidable, and can have a negative impact on the quality of the software~\cite{Lim2012Software}.

Due to the importance of technical debt, a number of studies empirically examined it and proposed techniques to enable its detection and management. The main findings of the prior work is that 1) there are different types of technical debt, e.g., defect debt, design debt, testing debt, and that design debt has the highest impact~\cite{Alves2014MTD,Marinescu2012IBM}; and 2) statically analyzing the source code can help detecting technical debt~\cite{Marinescu2004ICSM,Marinescu2010CSMR,Zazworka2013CSE}. In particular, these works use metric thresholds to detect code smells, which are considered as proxies for technical debt. 

One major drawback of using metrics to detect technical debt is that no one knows if the detected smells really constitute technical debt, or if they correspond to problems that the developers care about. Therefore, more recently, our work showed that using code comments can be effective in identifying self-admitted technical debt~\cite{Potdar2014ICSME}. This work uses comments to detect \emph{generic} technical debt, and did not focus on any specific type of technical debt.

\section{Motivating Example}
\label{sec:motivating_example}
As mentioned earlier, one of the first works on self-admitted technical debt was the work by Potdar and Shihab~\cite{Potdar2014ICSME}. Their work showed that it is possible to identify self-admitted technical debt using source code comments. However, in their work, Potdar and Shihab studied \textit{generic} technical debt, i.e., they did not discriminate between the different types of technical debt. For example, technical debt can be in the form of design debt, testing debt, defect debt, and documentation debt. 

Since our work focuses on \SADTD, we first examined the effectiveness of using the general comments used by Potdar and Shihab to detect design technical debt. We applied the comment patterns that we derived (which we present later in the paper) and the comment patterns from Potdar and Shihab on the studied open source projects. As expected, the results produced by the general comment patterns identified all types of technical debt, indicating the need for more specific comment patterns that can be used to effectively identify design technical debt.

To illustrate our point, we show some example comments flagged by Potdar and Shihab's approach in the first column of Table~\ref{tab:satdmotivation}. The second column of the table shows the comments that are detected by the comment patterns we propose in this paper, which focus on \SADTD. A comparison of the comments in Table~\ref{tab:satdmotivation} clearly shows that the more specific comment patterns detect design issues. 

This simple example shows that comment patterns that specifically target design technical debt are needed. Simply using the general comment patterns may yield unfavourable results. We elaborate more on the performance of using the general comment patterns to detect \SADTD in Section~\ref{sec:case_study_results}.

%Comparing the comments in the table \ref{tab:satdmotivation} with the comments in table \ref{tab:sadtdmotivation} it is clear  that the first approach would not make any distinction between different categories of technical debt like design debt. All of the comments listed is table \ref{tab:sadtdmotivation} are related with design debt in the other hand in table \ref{tab:satdmotivation} is not what happens. We also argue that identifying each specific category of technical debt is important since they are managed and corrected in different ways. 

\begin{table*}[!hbt]
    \begin{center}
        \caption{Example of General/Design Self-admitted Technical Debt Comments}
        \vspace{-2mm}
        \label{tab:satdmotivation}
        \begin{tabular}{ p{3in} | p{3in} } 
            \toprule
            \textbf{General Self-Admitted Technical Debt} &  \textbf{Self-Admitted Design Technical Debt}  \\ 
            \midrule
            \textit{remove this code once bug 62405 is fixed for the mainstream GTK} & \textit{This can lead to code smell, meh! Do we care}\\
            \textit{FIXME - This caching thing should not be here; it's brittle.} & \textit{This is an absurdly long method! Break it up.}\\
            \textit{FIXME compat: updateActionBars : should do something useful} & \textit{there should be an interface, instead of the         AbstractMessageFolder}\\
            \textit{FIXME this does not actually set the default since it is the wrong} & \textit{rethink where exactly some of the following methods belong (GenModel or GenPackage)}\\
            \textit{TODO: - please add some javadoc - ugly classname also} & \textit{Cyclic dependency with PersistenceManager}\\

            %\textit{FIXME: this is killing at least SSE editors, see bug 318034} & \textit{hack to support dockable view title update replace with listener pattern}\\
            %\textit{FIXME: this is not 64-bit clean} & \textit{This is in the wrong place.  It's not profile specific. It needs to be moved to main XMI reading code. }\\
            %\textit{HACK. Calling super.read() installs a required preferences change listener.} & \textit{This appears unused.  If it's needed, the Model API should be enhanced to provide a method that does this directly.}\\
            %\textit{This test is problematic. It makes assumptions about the behavior} & \textit{What does the magic number 6000 represent here? Put it in an explanatory literal! }\\
            %\textit{TODO: Won't our use of PathComparator take care of uniqueness?} & \textit{Downcast to avoid using an interface?  Yuck.}\\
            %\textit{FIXME: why override if nobody uses?} & \textit{We should actually rework this class to not implement Parser}\\
            %\textit{not exist yet. Throws a CoreException if there is a problem} & \textit{unhappy about this being public ... is there a better way?}\\
            %\textit{TODO this is such a hack it is silly.  There are still cases for race conditions etc} & \textit{remove use of instanceof!}\\
            %\textit{KLUDGE!! Test commented out until bug 170353 is fixed...} & \textit{a design flaw, it doesn't update properly}\\
            \bottomrule
        \end{tabular}
    \end{center}
\end{table*}

\section{Approach}
\label{sec:approach}
The main goal of our study is to extract comment patterns that can be used to effectively identify \SADTD. Figure~\ref{fig:approach} shows an overview of our approach. The following subsections detail each step of our approach.
 
\subsection{Data Extraction}

To perform our study, we obtain the source of ten large open source projects, namely Apache Ant, Jakarta Jmeter, ArgoUML,  Columba, EMF, Hibernate, JEdit, JFreeChart, JRuby and SQuirrel SQL Client. We chose the aforementioned projects, since they belong to different domains, and vary in size (e.g., LOC), and in the number of contributors.

Table~\ref{tab:projDetails} provides statistics about each of the projects used in our study. In total, we obtained more than 258,878 comments, found in 16,249 files. We also include the release used, the number of classes, and the total lines of code (LOC). In our study, we only use the Java files to calculate the LOC. It is important to notice that the number of comments shown for each project does not represent the number of commented lines, but rather the number of individual line, block, and Javadoc comments. 


%All the projects were downloaded from their respective websites, except from ArgoUml that we extracted the latest version,at the date, from its repository.  
%It represents the number of identified comments by the parsing tool. For example, a block comment of 10 lines will be treated as one comment, the same applies for Javadoc comments. It Does not matter how long a block or Javadoc comment can be, it will be always be counted as a single comment by the tool.

%Specific details for each one the projects are provided in table \ref{tab:projDetails}, including the release of each one of them, the number of classes, the number of comments and the number of lines of code. As a reference, we also added the number of contributors reported at OpenHub.net \cite{Openhub:home}. 

\begin{table*}[!tbh]
    \begin{center}
        \caption{Case Study Project Details and Statistics}
        \vspace{-2mm}
        \label{tab:projDetails}
        \begin{tabular}{l| c c c c c | p{2.5in}}
            \toprule
            \textbf{Project} & \textbf{Release} & \textbf{LOC} & \textbf{Classes} & \textbf{Comments} & \textbf{Contributors} & \textbf{Description}                                                      
            \\ \midrule
            Apache Ant       & 1.7.0            & 115,881      & 1,475            & 21,587            & 70                    & A Java library and command-line tool to build Java applications.           \\
            Jakarta Jmeter   & 2.3.2            & 81,307       & 1,181            & 20,084            & 32                    & An application to measure performance and assert functional behavior.      \\
            ArgoUML          & 0.34             & 176,839      & 2,609            & 67,716            & 87                    & An UML modeling tool.                                                      \\
            Columba          & 1.4              & 100,200      & 1,711            & 33,895            & 9                     & A desktop email client written in Java.                                    \\
            EMF              & 2.4.1            & 228,191      & 1,458            & 25,229            & 28                    & Eclipse Modeling Framework.                                                \\
            Hibernate        & 3.3.2 GA         & 173,467      & 1,356            & 11,630            & 216                   & An Object Relational Mapping framework.                                    \\
            JEdit            & 4.2              & 88,583       & 800              & 1,6991            & 55                    & A light weight text editor.                                                \\
            JFreeChart       & 1.0.19           & 132,296      & 1,065            & 23,123            & 18                    & A Java library to display graphics and charts.                             \\
            JRuby            & 1.4.0            & 150,060      & 1,486            & 11,149            & 291                   & Is the implementation of the Ruby language using the Java Virtual Machine. \\
            SQuirrel         & 3.0.3            & 215,234      & 3,108            & 27,474            & 40                    & A graphical SQL client written in Java.                                    
            \\ \bottomrule
        \end{tabular}
    \end{center}
\end{table*}

 
\subsection{Parse Source Code}

After obtaining the source code of all projects, we extract the comments from their source code. We use JDeodorant~\cite{Tsantalis2008CSMR}, an open-source Eclipse plug-in, to parse the source code and extract the code comments. Once extracted, we store all comments in a relational database to facilitate the processing of the data.

%keep this information in memory while it executes source code analysis. We took advantage of the parsing functionality and developed our own version of the tool, that stores the parsed information of the comments in the database. We list the number of classes found and number of comments for each project in table \ref{tab:projDetails}. In order parse the code properly, it is necessary that all analyzed projects can be compiled and built in the Eclipse environment. 

\subsection{Identification of \SADTD~Comment Patterns}
Once we store all comments in the database, our next step is to identify the \SADTD~comment patterns. Since we are dealing with natural language in the comments, it is challenging to automatically determine what comments indicate design technical debt. Therefore, we opted to use two different approaches to determine comment patterns that indicate design technical debt. First, we use the terms mentioned in prior work~\cite{fowler1999refactoring,brown1998antipatterns,martin2009clean} (i.e., code smell and anti-pattern names) as indicators of design problems to determine comments that are indicative of design technical debt. Second, we manually examined and classified all  comments of one project i.e., Apache Ant, in order to determine comment patterns that are indicative of \SADTD. After analyzing the results, we found that combining comment patterns from the two aforementioned approaches provides the best results. We detail the steps taken to achieve each of the two approaches.

%\noindent \textbf{Using well-known terms to identify \SADTD comments}
%
%To create the Design Technical Debt patterns we took advantage of the design flaws names that can be found in Fowler's book \cite{fowler1999refactoring}, Brown et al book \cite{brown1998antipatterns} and Martin's book \cite{martin2009clean}. Based on the words extracted from these books and our experience, we searched for synonyms in the database. The first step is verify if the selected word has a match in the database, then we read all the comments related to it. Doing so, we were able to identify new words and more than that, understand the frequency that this word is used as \SADTD. 
%Following this approach we were able to come up with three different dictionaries for patterns to find \SADTD comments. 
%  
%The first group of patterns we called the ``-ilities'' patterns. It possesses words like ``Configurability'', ``Security'', etc. It contains 17 words and in our preliminary analysis we matched 181 comments. After a quick inspection we kept 14 of these comments as potential \SADTD comments. The following are examples of the kept comments: \textit{ ``Apparently in some environments you can't catch the security exception at all... will probably have to work around''} and \textit{``pretty weak and don't provide real security.''}.
% 
%We based the second group of patterns in the names of well-know bad smells and anti-patterns like ``Clone'',``Dead'', etc. Using these patterns we found the following potential  \SADTD comments in the source code: \textit{``this class is considered to be dead code by the Ant developers and is unmaintained. Don't use it''}. The anti-patterns patterns contains 8 words, in our preliminary analysis we matched 1,316 comments and after a quick inspection we kept 148 of these comments.
%
%The last group of patterns we put the words that can be related with Design such as: ``Ambiguous'', ``Avoid'', ``Big'', etc. We found 3,449 matches using 14 different words. Of those we kept 196. The following are examples of the kept comments:\textit{``...find a way to avoid the cost of creating a String here''}. 
%
%At the end of this step, we were able to come up with three different dictionaries for patterns to find \SADTD comments.
%
%To test the performance of the dictionaries we have quantified the number of comments that matches with any of the words contained in one of the dictionaries, then we did a quick manual examination to inspect the results obtained. The manual inspection done in this phase took about 8 hours. 
%
%Using the three group of patterns we matched 4,946 comments. Out of that we classified 358 comments as potential \SADTD comments.
%To classify the comment as a potential \SADTD comment we read trough the 4,946 comments where patterns matched the words in the comment and then we  eliminated the matches that clearly did not represent a \SADTD. 
%
%During this analysis we notice that the number of matches is way higher than the number of \SADTD. Which means that we are obtaining a high number of false positives. To mitigate this problem we took action in two dimensions: the comments and the \SADTD patterns. First, we need to eliminate comments that are irrelevant and second, we need to improve our \SADTD patterns to be more precise. 

\subsubsection{Applying Heuristics to Eliminate Irrelevant Comments}

When applying our first approach, i.e., using the terms in the prior work to identify comments that are indicative of \SADTD, we found that we are able to flag comments that indicate design issues, but also flag many false positives. We analyzed the false positives to see whether we can gain any insight into why they appear and how we can eliminate them. 

We identified three main types of false positives. First, license comments, containing copyright information and legal rights. Second, commented source code containing Java keywords, e.g., ``big'' and ``long''. Finally, Javadoc comments were flagged, however, they often had no relation to design issues. As a result, we came up with three heuristics and a post-processing step to reduce the number of false positives.

\begin{table*}[!hbt]
    \begin{center}
        \caption{Number of Comments After the Application of Each Heuristic}
        \vspace{-2mm}
        \label{tab:heuristicDetails}
        \begin{tabular}{l| p{.6in} p{.6in} p{.8in} p{.7in} p{.55in}} 
            \toprule
            \textbf{Project} &  \textbf{Initial no. of Comments} & \textbf{After license heuristic} &  \textbf{After comment code heuristic}  &  \textbf{After Javadoc heuristic} & \textbf{After post processing} \\ 
            \midrule
            Apache Ant & 21,587 & 20,421 & 20,268 & 6,239 & 4,436 \\ 
            Jakarta Jmeter & 20,084& 18,840 & 18,530 & 12,360 & 8,126 \\
            ArgoUML & 67,716 & 28,180 & 27,848 & 13,972 & 10,303 \\
            Columba & 33,895 & 14,600 & 14,256 & 9,095 & 6,825 \\
            EMF & 25,229 & 24,355 & 24,093 & 8,861 & 5,868 \\
            Hibernate  & 11,630 & 10,446 & 10,277 & 4,908 & 3,071 \\
            JEdit & 16,991 & 16,128 & 16,037 & 13,118 & 11,232 \\
            JFreeChart & 23,123 & 22,114 & 22,047 & 5,902 & 4,449 \\
            JRuby & 11,149 & 10,274 & 10,080 & 6,887 & 5,176 \\
            SQuirrel  & 27,474& 25,566 & 25,196 & 13,713 & 8,627 \\  
            \bottomrule
        \end{tabular}
    \end{center}
\end{table*}
  
 
\begin{itemize}


\item{\textbf{Heuristic to remove license comments.}} 
When license comments are added to the Java files in a project they are generally placed in the first lines of the file, before the class declaration. Based on this knowledge we created a heuristic that eliminates comments that are placed before the class declaration. To validate the result of this heuristic we examined a sample of the comments being removed to check if they were indeed license comments. We noticed that some comments were placed before the class declaration although they were not license comments. To mitigate the risk of eliminating important comments, we added one more condition: If the comment contains one of the task-reserved words (e.g. ``todo'', ``fixme'', or ``xxx'') we do not remove the comment.

\item{\textbf{Heuristic to remove commented source code.}}
If a commented piece of source code contains Java keywords like ``long'' or ``big'', it will increase the number of false positives of our approach. Commented source code can be found for several different reasons. One of the possibilities could be that the code is not being currently used, or if the particular piece of code is used to debug the program. Since commented code does not have \SADTD, we remove commented source code using a regular expressions that captures typical Java code structures.

\item{\textbf{Heuristic to remove Javadoc comments.}}
The Javadoc comments contain information about the purpose and use of methods and classes. That said, Javadoc comments rarely mention \SADTD. Therefore, we create a heuristic that removes Javadoc comments. To mitigate the risk of eliminating some correct cases, we added one exception - if the comment contains one of the task-reserved words (e.g. ``todo'', ``fixme'', or ``xxx'') we keep that Javadoc comment. 

%We argue that since these comments appear in the public documentation of projects, it is less likely that a developer will add a \SADTD~comment there. 
\item{\textbf{Post processing technique to merge multiple line comments}}
Another problem that we found while analyzing the comments was that some times developers make long comments, using multiple single-line comments instead of a Block comment. Treating every single line of a long comment as an individual comment causes us to miss important context details that could be recovered by treating all single-line comments as a single block comment. Therefore, we create a post processing technique that searches for consecutive single-line comments and groups them. 

\end{itemize}

The steps mentioned above significantly reduced the number of comments in our dataset and helped us focus on the most applicable and insightful comments. For example, in the Apache Ant project, applying the above steps helped reduce the number of comments from 21,587 to 4,436 comments.


\subsubsection{Manual investigation of identified \SADTD~comments} 
%Once we understand the pattens in the one dataset, we can add then to our \SADTD patterns. The first necessary step to do that is to create the dataset, as to the best of our knowledge, there is none available to the date. We choose one of the projects to manually classify all comments and create this dataset, the selected project was Apache Ant. 

In addition to using the words that indicate design issues to detect \SADTD, we also manually examine our dataset to extract comment patterns that indicate~\SADTD~comments. We started by examining all of the 4,436 comments for the Apache Ant project and classified each comment as being related to \SADTD or not. Since our focus in this work is on design debt, comments related to other types of technical debt were not labeled as \SADTD comments. The classification of the Apache Ant comments took approximately 32 hours and was performed by the first author of the paper.

\noindent \textbf{Manual Examination of Comments to Identify \SADTD~Comment Patterns}

In the end of the classification we identified 93 \SADTD~related comments out of 4,436 comments in Apache Ant project.

Our next goal was to abstract the comments and come up with a set of \emph{comment patterns} that indicate \SADTD. Comment patterns are general patterns that represent one or more comments. Simply using a single word to identify \SADTD~comments can be misleading since the context that the word appears in can completely change the meaning of that word. In order to address this issue, we take into consideration some of the other words that appear in the same sentence to combine them into what we call comment patterns. 

%To create comment patterns, we combine multiple \SADTD~words, so that some context is considered.

% with word task indicators ones, (e.g, ``todo'', ``fixme'', ``xxx'') in addition to the expressions that we were creating observing the \SADTD comments in the dataset. 

%For every new \SADTD comment pattern that was created we manually sample the results in two different databases, one had all the comments for Apache Ant and the other one had all the comments for all the remaining projects. While exploring the results we notice that small variations in the expression used to find  \SADTD comments can improve the results. For example we apply the patterns ``place some where else'' and ``move somewhere else'' to identify misplaced code.

By the end of this step, we had identified the comment patterns that indicate \SADTD. \textbf{In total, we had 176 comment patterns that can be used to detect \SADTD}. To facilitate future work in the area, we make our dataset and the comment patterns publicly available \footnote{http://users.encs.concordia.ca/~e\_silvam/publications.html}. 

Table \ref{tab:dictionarySample} provides a sample of the comment patterns that we used to identify \SADTD~comments. 
%The first column of Table~\ref{tab:dictionarySample} shows  the type of design issue and the second column shows the \SADTD~ comment pattern capturing the corresponding design issue. 
The `\%' symbol indicates that the pattern uses the SQL language wildcards. Wildcards make the query to match anything before or after the wildcard symbol. For example, ``dependen\%'' would result in positive results for comments that contains the words ``dependency'' or ``dependencies''.

\begin{table}[t!]
    \begin{center}
        \caption{Sample \SADTD Comment Patterns}
        \vspace{-2mm}
        \label{tab:dictionarySample}
        \begin{tabular}{ c }
            \toprule
            \textbf{Related Comment Patterns} \\ 
            \midrule
                 `\%future\%may\%'       \\
                 `\%future\%better\%'  \\
                 `\%future\%enhance\%' \\ 
                 `\%future\%change\%'  \\   
             `\%dependency\%cycle\%'  \\
             `\%todo\%dependenc\%'    \\
             `\%fixme\%dependenc\%'   \\
             `\%xxx\%dependenc\%'    \\
            \bottomrule             
        \end{tabular}
    \end{center}    
\end{table}


Once we derive the 176 comment patterns that indicate \SADTD, we use these patterns to answer our research questions, which we detail in the next section.


%\subsection{Applying Comment Patterns and Measuring their Performance}
%\label{sec:applying_comment_patterns_measuring_performance}
%We conduct an experiment to measure the performance of our approach using precision and recall. \emph{Precision} measures how many of the comments flagged using our comment patterns are indeed \SADTD. \emph{Recall} measures how many of the comments indicating \SADTD~our approach can catch. To measure recall we first needed to classify a dataset and labeling all of the \SADTD~comments of the project. Since measuring recall is a difficult and time consuming task, we report recall values for three projects, Apache Ant, Jakarta Jmeter and JFreeChart.
%
%We measure first the results of the three first dictionaries, ``-ilities'', ``design'' and ``bad smells''. As our first classified dataset was Apache Ant, we use it to measure precision and recall of the dictionaries. 

%Table \ref{tab:dictionaryEvaluation} shows the results for each one of the dictionaries found analyzing the Apache Ant dataset, which has 93 \SADTD comments. When running the ``-ilities'' patterns we found 10 matches of which 2 were classified  as \SADTD comments. That represents a precision of 20\% and recall of 2.15\%.  Using the Design patterns we found 54 matches of which 5 were classified  as \SADTD comments. For the Bad Smell patterns we found 39 matches and 1 \SADTD comment. The precision and recall for these dictionaries were 9.26\% , 5.38\% and 2.56\% , 1.01\% respectively. 

%The performance of the three dictionaries was not in the desired level yet, and as we analyze the \SADTD comments from Apache Ant we notice that expressions patterns represents better our desired dataset than single words patterns. We created then expressions for the three dictionaries combining their words with task specific ones like ``todo'',``fixme'' and ``xxx''. 

%As a result we come up with a unified expression group of patterns that have expressions based on the three previous group of patterns and expressions based on the \SADTD comments of Apache Ant project. The performance of this new group of pattern can be found in details in Table \ref{tab:expressiondictionary_precision} and Table \ref{tab:expressiondictionary_recall}. For Apache Ant we got 96.30\% precision and 83.87\% recall. For Jakarta Jmeter we found 75 matches of that 66 were \SADTD comments, the recall as of 27.16\%. JFreeChart we found 12 matches out of that 10 was classified as \SADTD comments. The recall for JFreeChart was 10.87\%.

%Finally we measured performance for the other seven projects remaining to be analyzed. As it is necessary to classify the dataset to measure recall, and this task is very time consuming, we did not classify them due time constrains. The precision achieved considering the comments of all projects was of 85.86\%. We found 964 comments out of 825 \SADTD comments.

\section{Case study Results}
\label{sec:case_study_results}
The goal of our research is to develop an automatic way to detect design and requirement \SATD comments \emad{need to make sure this is consistent with the goal in the beg of section 2}. To do so, we first manually classify a large number of comments identifying which ones are \SATD. With the resulting dataset, we train the NLP tool to identify design and requirement \SATD (RQ1). To better understand what words indicate \SATD, we inspect the features used by the NLP tool to identify the detected technical debt. These features are words that are frequently found in comments with technical debt. We present the 10 most common words that indicate design and requirement \SATD (RQ2). Since the manual classification required to create our training dataset is expensive, ideally we would like to achieve maximum performance with the least amount of training data. Therefore, we investigate how variations in the amount of training data affects the performance of our classification (RQ3). We detail the motivation, approach and present the results of each of our research questions in the remainder of this section.    

\begin{figure*}[!thb]
  \centering
  % \subfigure[Design Debt]{\includegraphics[width=0.48\textwidth]{figures/f1_measure_comparisom_design.pdf}
  \subfigure[Design Debt]{\includegraphics[width=0.4\textwidth]{figures/f1_measure_comparisom_design_1.pdf}
  \label{fig:f1_measure_comparison_design_debt}}
  % \subfigure[Requirement Debt]{\includegraphics[width=0.48\textwidth]{figures/f1_measure_comparisom_requirement.pdf}
  \subfigure[Requirement Debt]{\includegraphics[width=0.4\textwidth]{figures/f1_measure_comparisom_requirement_1.pdf}
  \label{fig:f1_measure_comparison_requirement_debt}}
  \caption{F1 measure comparison}
\end{figure*}


\begin{table*}[!thb]
    \begin{center}
        \caption{Improvement over the baselines F1 measure for design and requirement debt}
        \label{tbl:improvement_f1measure}
        \begin{tabular}{l| c c c c c| c c c}
        \toprule
        
        % draw first line. The * centralizes the Project column, then set the total size of columns that we have
        \multirow{4}{*}{\textbf{\thead{Project}}} & \multicolumn{5}{c|}{\textbf{\thead{Design debt}}} & \multicolumn{3}{c}{\textbf{\thead{Requirement debt}}} 
        % indicates that from now on we are filling the content of the next line
        \\ 

        % remainder columns
        & {\textbf{\thead{Our\\approach}}} & {\textbf{\thead{Comment\\patterns}}} & {\textbf{\thead{Random\\classifier}}} & {\textbf{\thead{Improvement over\\comment patterns}}} & {\textbf{\thead{Improvement over\\random classifier}}} & {\textbf{\thead{Our\\approach}}} & {\textbf{\thead{Random\\classifier}}} & {\textbf{\thead{Improvement over\\random classifier}}} \\

        
        
        \midrule                                                  
        \textbf{Ant}          &  0.517  &  0.175 & 0.045 &  2.9$\times$   &  11.4$\times$  & 0.154  &  0.006  & 25.6$\times$  \\
        \textbf{ArgoUML}      &  0.814  &  0.078 & 0.155 &  10.4$\times$  &  5.2$\times$   & 0.595  &  0.083  & 7.1$\times$   \\
        \textbf{Columba}      &  0.601  &  0.145 & 0.038 &  4.1$\times$   &  15.8$\times$  & 0.804  &  0.013  & 61.8$\times$  \\
        \textbf{EMF}          &  0.470  &  0.114 & 0.035 &  4.1$\times$   &  13.4$\times$  & 0.381  &  0.007  & 54.4$\times$  \\
        \textbf{Hibernate}    &  0.744  &   0.15 & 0.214 &  4.9$\times$   &  3.4$\times$   & 0.476  &  0.042  & 11.3$\times$  \\
        \textbf{JEdit}        &  0.509  &  0.324 & 0.037 &  1.5$\times$   &  13.7$\times$  & 0.091  &  0.003  & 30.3$\times$  \\
        \textbf{JFreeChart}   &  0.492  &  0.053 &  0.08 &  9.2$\times$   &  6.1$\times$   & 0.321  &  0.007  & 45.8$\times$  \\
        \textbf{Jmeter}       &  0.731  &  0.127 & 0.075 &  5.7$\times$   &  9.7$\times$   & 0.237  &  0.005  & 47.4$\times$  \\
        \textbf{JRuby}        &  0.783  &  0.138 & 0.131 &  5.6$\times$   &  5.9$\times$   & 0.435  &  0.044  & 9.8$\times$   \\
        \textbf{SQuirrel}     &  0.540  &  0.071 & 0.056 &  7.6$\times$   &  9.6$\times$   & 0.541  &  0.014  & 38.6$\times$  \\
        \midrule 
        \textbf{Average}      &  0.620  & 0.137 & 0.086 &   4.5$\times$   & 7.1$\times$     & 0.403  &  0.022  & 18$\times$  \\ 
        \bottomrule
        \end{tabular}
    \end{center}    
\end{table*}


\vspace{3mm}
\noindent\rqi
\vspace{3mm}

\noindent \textbf{Motivation:} As shown in previous work \cite{Potdar2014ICSME, Maldonado2015MTD}, \SATD comments can be found in the source code. However, there is no automatic way to identify these technical debt comments. The methods proposed so far heavily rely on manual examination of source code, and there is no evidence on how well these approaches perform. Moreover, the state-of-the-art approaches to detect \SATD do not discriminate between the different types of technical debt (e.g., design, test, requirements).

Therefore, we want to determine if NLP tools such as, the Stanford NLP Classifier, can help us to surpass these limitations. NLP tools can automatically classify comments based on specific linguistic characteristics of these comments. Answering this question is important since it helps us understand the opportunities and limitations of using NLP techniques to automatic identify \SATD comments. 


%These characteristics are obtained through the training dataset that we created. 

%For example, the training dataset can show that the adjective `ugly' is frequently found in \SATD comments. 

\vspace{1mm}
\noindent \textbf{Approach:} In this RQ, we would like to examine how effectively we can identify design and requirement \SATD. Therefore, the first step is to create a dataset that we can train and test our NLP classifier on. We classified the source code comments into the following types of \SATD: design debt, defect debt, documentation debt, requirement debt and test debt. However, our previous work showed that the most frequent \SATD comments are design and requirement debt. Therefore, in this paper, we focus our attention on the identification of these two types of \SATD. We decided to focus on these two types of \SATD since 1) they are the more common types of technical debt and 2) since our approach is NLP-based and the NLP tools require sufficient data to train their models (i.e., we can not build a model with 10 or 15 samples).

%As described in Section \ref{sec:approach}, in a combined effort between this work and our previous study on \SATD \cite{Maldonado2015MTD} we manually classify comments from ten open source projects. 



We train the NLP classifier using our manually created dataset. The dataset contains comments with and without \SATD, and each comment contains its own classification (i.e., without technical debt, design debt or requirement debt). Then, we add to the training dataset all comments classified as without technical debt and the comments classified as the specific type of \SATD that we want to detect/predict. We use comments from 9 out of 10 projects that we analyzed to create the training dataset. The comments from the remaining one project is used to evaluate the classification performed by the NLP tool. We choose to create the training dataset using comments from 9 out of 10 projects since we want to train the NLP tool with the most diverse data possible (i.e., comments from different domains of applications). However, we discuss the implications of using different amounts of training data in RQ3. We repeat this process for each of the ten projects, each time training on 9 projects and testing on the remaining 1 project.

Based on the training dataset, the NLP tool will classify each comment in the test dataset. The resulting classification is compared with the manual classification provided in the test dataset and evaluated. If a comment in the test dataset has the same classification as the classification suggested by the NLP tool, we will have a true positive (tp) or a true negative (tn). \everton{True positives are the cases where the NLP tool correctly identifies \SATD comments, and true negatives are comments without technical debt that are classified as without technical debt}. Similarly, when the classification provided by the tool diverges from the classification provided in the test dataset we have false positives or false negatives. False positives (fp) are comments classified as being \SATD when they are not, and false negatives (fn) are comments classified as without technical debt when they really are \SATD comments. Using the tp, tn, fp, and fn, we are able to evaluate the performance of the different detection approaches in terms of precision (i.e., $\frac{tp}{tp + fp}$), recall (i.e., $\frac{tp}{tp + fn}$) and F1 measure (i.e., $2 \times \frac{P \times R}{P + R}$). To determine how effective the NLP classification is, we compare its F1 measure with the F1 measure of two other approaches. We use the F1 measure to compare the performance between the approaches as this measurement provides the harmonic mean of precision and recall. Using the F1 measure allows us to incorporate the tradeoff between precision and recall and present one value that evaluates both measures.


We compare the performance of our NLP-based appraoch to two other approaches. \everton{the previous statement is necessary ?}The first approach is the current state-of-the-art in detection of \SATD comments devised by Potdar and Shihab ~\cite{Potdar2014ICSME}. This approach uses 62 comment patterns (i.e., keywords) that were noticed as recurrent in \SATD comments during the manual inspection of 101,762 comments. The second approach is the simple (random) baseline, which assumes that the detection of \SATD is random. The precision of this approach is calculated by taking the total number of \SATD over the total number of comments of each project. For example, Ant has 4,137 comments, of those, only 95 comments are design \SATD. The chance of randomly finding a design \SATD comment is 0.022 (i.e., $\frac{95}{4,137}$). Similarly, to calculate the recall we take into consideration the two possible classifications available: one is the type of \SATD (e.g., design) and without technical debt. Therefore, there is a 50\% chance that the comment will be a \SATD. 

\vspace{1mm}

\noindent \textbf{Results - design debt:} Table \ref{tbl:improvement_f1measure} presents the F1 measure of the three approaches, as well as the improvement achieved by our appraoch compared to the two compared approaches. We see that for all the projects, the F1 measure achieved by our approach is higher than the other baseline's F1 measures. The F1 measure obtained by our NLP-based appraoch ranges between 0.470 - 0.814, with an average of 0.620. In comparison, the F1 measure using the comment patterns ranges between 0.071 - 0.324, with an average of 0.137, and the simple (random) baseline classifier achieves F1 measures in the range of 0.035 - 0.214, with an average of 0.086. Figure \ref{fig:f1_measure_comparison_design_debt} visualizes the comparison of the F1 measure for our approach, the comment patterns approach and the simple (random) baseline classifier. We see from both, Table \ref{tbl:improvement_f1measure} and Figure \ref{fig:f1_measure_comparison_design_debt} that on average, our approach to identify design \SATD outperforms, the state-of-the-art comment pattern approach by 4.5 times and the simple (random) baseline approach by a factor of 7.1 times. 

It is important to note that each one of the selected baselines for comparison has one strong point. The comment patterns approach has a high precision, but it lacks recall, i.e., this approach points correctly to \SATD comments, but as the approach depends on keywords, it identifies a very small subset of all the \SATD comments in the project. 

%The less sophisticated random classifier baseline provides a high recall because we try to classify comments between two categories (i.e with or without technical debt), meaning a 50\% chance to randomly get one of the two classifications. 


%The second, third and fourth columns of Table \ref{tbl:improvement_f1measure} shows our approach F1 measure, the comment patterns baseline F1 measure and the random classifier baseline F1 measure respectively for each project while classifying design \SATD. In addition, fifth and sixth columns present how many times our approach surpass the comment patterns baseline and the random classifier baseline.   
 
\noindent \textbf{Results - requirement debt:}  The last three columns of Table \ref{tbl:improvement_f1measure} presents the performance of our approach, the performance of the simple (random) baseline approach and the improvement of our approach over the simple (random) baseline. The comment patterns appraoch was not able to identify an requirement \SATD, hence we omit the presentation of its results. For all projects, the F1 measure obtained by our approach surpass the simple (random) baseline classifier. Our appraoch achieves a F1 measure between 0.091 - 0.804 with an average of 0.403, whereas the simple (random) baseline achieves F1 measure in the range of 0.003 - 0.083, with an average of 0.022. Figure \ref{fig:f1_measure_comparison_requirement_debt} visualizes the performance comparison of the two approaches.

Generally, it is clear that requirement \SATD is less common than design \SATD, which makes it more difficult to detect. That said, our NLP-base approach provides a significant improvement over the simple (random) baseline appraoch, with an average improvement of 18 times.

%Despite the fact that this number is slightly lower than the average obtained while identifying design debt, the quantity of requirement \SATD comments that are present in the projects is very small, which makes its classification more difficult. For example, JEdit has 14 requirement \SATD comments distributed over a total of 10,080 comments (i.e, requirement debt comments plus without \SATD comments). Nevertheless, our approach can identify \SATD even in this unbalanced dataset as we still outperform the random classifier baseline F1 measured by 30.3 times, i.e., the random classifier baseline F1 measure was of 0.003 for JEdit. 

%highest F1 measure was obtained for Columba project with 0.804, and the lowest value was of 0.091 on JEdit. Although there was a big fluctuation between the maximum and minimum value obtained our average F1 measure was of 0.403. 

%Similarly, Figure \ref{fig:f1_measure_comparison_requirement_debt} shows the comparison between the three F1 measures while identifying requirement \SATD.

%We find that the comment patterns approach is not an appropriate approach for identifying requirement \SATD, since it was not able to identify any of the requirement \SATD comments in our analyzed projects. Therefore, we use only the random classifier baseline to compare with our results.





%The seventh column Table \ref{tbl:improvement_f1measure} shows the F1 measure of our approach, the eighth column presents the random classifier baseline F1 measure, and the ninth column shows the improvement of our approach over the random classifier baseline. For example, on SQuirrel project the F1 measure was of 0.541 whereas the random classifier baseline F1 measure was of 0.014, which means that there was an improvement of 38.6 times in the identification of requirement \SATD comments using our approach. 

% The eighth column of Table \ref{tbl:improvement_f1measure} shows the improvement of our approach over the random classifier baseline. For example, on SQuirrel project the F1 measure was of 0.875 whereas the random classifier baseline F1 measure was of 0.04, which means that there was an improvement of 21.8 times in the identification of requirement \SATD comments using our approach. 

\conclusionbox{We find that our NLP-based approach, is more effective in identifying \SATD comments compared to the the current state-of-art approaches. We achieved an average F1 measure of 0.620 when identifying design debt (an average improvement of 4.5X and 7.1X compared to other approaches) and an average F1 measure of 0.403 when identifying requirement debt (an average improvement of 18X).}

\vspace{3mm}
\noindent\rqii
\vspace{3mm}

\noindent \textbf{Motivation:} After asserting the efficiency of our NLP-based appraoch in identifying \SATD comments we want to better understand what words developers use when indicating this technical debt. Answering this question will provide insightful information that can guide future research direction, broaden our understanding on \SATD and also help us to detect it.     

\vspace{1mm}
\noindent \textbf{Approach:} To perform its detection, the NLP tool learns optimal features that can be used to detect \SATD. These features, are fragments of data (e.g., words) that are associated with a specific type of debt (e.g., design debt, requirement debt or without technical debt). Moreover, each of the features has a weight, which represents how strongly the feature relates to a specific type of debt. The NLP tool uses the labeled training data that we input to determine the features and their weigh. Then, the features and their corresponding weights are used to determine if a comment belongs to a specific type of \SATD.

For example, if the NLP tools, based on the training data, determines that the two features ``hack'' and ``dirty'' are related to design debt with weight 5.3 and 3.2, respectively, and the feature ``something'' relates to the non-technical debt class with a weight of 4.1. Then, if we aim to classify the comment ``this is a dirty hack it's better do to something'' in our test data, all features will be analyzed and the following score would be calculated design debt weight = 8.5 (i.e., feature `hack' weight plus feature `dirty' weight) and without technical debt weight = 4.1 resulting in a comment classified as design debt.

%The features are extracted from the comments in the training dataset, and then applied to the test dataset where they are combined to reach a vote. That is, every feature that is satisfied by the comment being classified (i.e., matched) will be used to predict the class for the comment. The vote is given to the class with highest weight. 

For each analyzed project, we collect the features used to predict the \SATD comments. These features are provided by the NLP tool as output and stored in a text file. The features are written in the file based on the weight that they have, ordered by highest weight to the lowest weight, meaning more relevant features to less relevant features respectively. Based on these files, we rank the words calculating the average ranking position of the analyzed feature across the ten different projects. 

%We determined the top 10 features (i.e, most relevant based on the weight) for design and requirement \SATD comments.

\noindent \textbf{Results:} Table \ref{tbl:top_ten_features} shows the top 10 textual features used to the identify  \SATD in the ten studied projects, ordered by their weight \emad{relevance or weight?}. The first column we present the ranking of each textual features, the second column lists the features used in the identification of \emph{design} \SATD, and the third column lists the textual features used to identify \emph{requirement} \SATD.

\begin{table}[!thb]
    \begin{center}
        \caption{Top Ten Textual Features Used to Identify Design and Requirement Self-Admitted Technical Debt}
        \label{tbl:top_ten_features}
        \begin{tabular}{l| l l }
        \toprule
        \textbf{Rank} & \textbf{Design Debt} & \textbf{Requirement Debt}  \\
        \midrule
         1  & hack       &   todo              \\
         2  & workaround &   needed            \\
         3  & yuck!      &   implementation    \\
         4  & kludge     &   fixme             \\
         5  & stupidity  &   xxx               \\
         6  & needed?    &   ends?             \\
         7  & columns?   &   convention        \\
         8  & unused?    &   configurable      \\
         9  & wtf?       &   apparently        \\
         10 & todo       &   fudging           \\
        \bottomrule
        \end{tabular}
    \end{center}    
\end{table}

From Table~\ref{tbl:top_ten_features} we observe that top ranked textual features, i.e., hack, workaround, yuck!, kludge and stupidity, are related to design \SATD indicate sloppy or mediocre source code. Other textual features such as needed, columns?, unused, wtf? and todo, are questioning the usefulness or utility of specific source code. For requirement \SATD, the top ranked features, i.e., todo, needed, implementation, and fixme, indicate the need to enhance or complete the implementation in the future. Other lower ranked textual features such as xxx, ends?, convention, configurable, apparently and fudging, indicates potential future enhancements that makes the code more configurable and/or generic \emad{not sure if I am using the right words here - maybe even add 1 or 2 examples here}.

We also observe that it is possible a single textual feature can be used to indicate design and requirement \SATD. However, in such cases, the ranking of the textual features in for design and requirement \SATD is different. For example, the work todo is ranked 10 for design debt, whereas it is ranked first for requirement debt. This finding makes intuitive sense, since requirement debt will naturally be related to the implementation of future functionality.

It is important to note here that although we present the top 10 textual features, the classification of the comments is based on a combination of several textual features. In fact, on average, the number of features to classify design debt is 6,196 and 2,889 for requirement debt. The exact number of unique textual features used to detect \SATD for each project in shown in Tabel \emad{Everton add table}. The fact that opur NLP-based approarch leverages so many features helps to explain the significant improvement we are able to achieve over the state-of-the-art~\cite{Potdar2014ICSME}, which only uses 62 patterns that are composed of \todo{X} unique textual features.

%This shows why our recall is much better than Potdar and Shihab's ~\cite{Potdar2014ICSME}. 

\conclusionbox{We find the that design and requirement debt have their own textual features that best indicate such \SATD comments. For design debt, the best indicative textual features indicate sloppy or mediocre source code, whereas for requirement debt they relate to the need to enhance or complete the implementation in the future.}

\vspace{3mm}
\noindent\rqiii
\vspace{3mm}

\noindent \textbf{Motivation:} Thus far, our we have shown that our NLP-based approach can effectively identify \SATD comments. However, we conjuncture that the performance of the classification depends on the amount of training data. At the same time, creating the training dataset is expensive and manually intensive. So, the question that arises is: how much training data do we need to effectively classify the \SATD comments? If we need a high number of comments to create our training dataset, our approach will be more difficult to extend and applied for other projects. On the other hand, if a small dataset can be used to identify \SATD comments then this approach can be applied with minimal effort, i.e., less training data. That said, intuitively we expect that the performance of the classifier would improve with each iteration as more comments are being added to the training dataset.


\noindent \textbf{Approach:} To answer our research question, we followed a systematic process where we incrementally add training data and evaluate the performance of the classification. Since we have 10 projects in our dataset, we use one project as testing data, and the remaining nine projects to train. However, we do not train on all nine projects, instead, we add each project incrementally. We repeated this process for each project and report on our findings.

%We executed the classification process several times with an increasing number of comments being added to the training dataset while collecting the results to analyze the changes between each iteration.

%We first select a project that will be classified by the Stanford Classifier.

%Second, with one of the remainder projects, we select all \SATD comments (i.e., design or requirement debt accordingly with what we want to classify), and all comments without technical debt as well. These comments are added to the training dataset and fed into the Standford Classifier. Then, the results of the classification are collected and stored for analysis.

%Third, the comments of other project is added to the training dataset and the results of the classification are collected once more. The order that projects are added to the training dataset is not random. We add first projects which has more \SATD comments of the specific type that we are trying to identify. We cycle through the above steps until we had added the comments of all nine projects to the training dataset.

%Fourth, we select another project to be classified by the Stanford Classifier, and we repeat this process until we have analyzed all our projects.

To determine how much data is required to effectively identify \SATD comments, we determine F1-measure after each iteration (an iteration is simply a run with different sizes of training data). We record the iteration that achieves the highest F1-measure and the number of projects used in the training dataset to achieve this F1-measure. Then, we record the number of projects need to achieve at least 90\% and 80\% of the maximum F1-measure.
%Based on the maximum F1 measure we calculate how close the others iterations were from achieving the same value.

For example, if the maximum F1-measure is 0.85 and is achieved in the 8th iteration (i.e., using 8 projects in the training dataset), and during the 4th iteration we achieve a F1-measure of 0.80. Then, we say that we can achieve at least 90\% (94\% to be exact) of the maximum F1-measure with 4 projects. Since the results will differ for the different projects, we repeat this analysis for all projects and present the averages across all projects.
% Once the percentage of the maximum F1 measure is calculated for all iterations we analyze the quantity of comments used in the training dataset by them.

\begin{figure}[t]
  \centering
  \includegraphics[width = 0.48\textwidth]{figures/design_ant.pdf}
  \vspace{-3mm}
  \caption{Ant Design Debt classification}
  \label{fig:design_ant_result}
\end{figure}

\begin{figure}[t]
  \centering
  \includegraphics[width = 0.48\textwidth]{figures/implementation_argo}
  \vspace{-3mm}
  \caption{ArgoUml Requirement Debt classification}
  \label{fig:implementation_argo_result}
\end{figure}




%Therefore, we determined which iteration has the highest performance for each project. We called this iteration as the maximum F1 measure. To know how close the other iterations were from reaching the maximum performance we use the iteration F1 measure value to calculate the percentage of the maximum F1 measure that it represents. We calculate these values for design and requirement debt separately.

\noindent \textbf{Results - design debt:}  Figure~\ref{fig:design_ant_result} shows the F1-measure using different amounts of training data for the Ant project. Due to space, we discuss the results for a representative project (Ant) in this section, however, figure for all projects are provided in the Appendix \todo{add}. From Figure~\ref{fig:design_ant_result}, we find that the maximum F1-measure improves as we increase the number of projects (i.e., iterations), achieving the highest F1-measure in the seventh iteration and slightly decreasing afterwards. The horizontal lines in the figure show the 90\% and 80\% of the highest F1-measure. We can see from Figure~\ref{fig:design_ant_result} that with \todo{1,499 comments (i.e., from 3 projects)} and \todo{x}, we can achieve 90\% and 80\%of the highest F1-measure, respectively. This amounts to a reduction in \todo{X\%} in training data to achieve 90\% and 80\% of the maximum F1-measure, respectively. Considering the tradeoff in accuracy versus the amount of training data, for Ant, using only \todo{3} or \todo{4} projects provides the best tradeoff.

%The three highest values of the F1 measure in Figure \ref{fig:design_ant_result} are: 0.526, 0.539 and 0.557 obtained during the 6th, 8th and the 7th iteration respectively.

\emad{Everton, copy the correct results from the commented out paragraph here. Also add the horizontal lines to the figures.}
%We notice that in the 3rd iteration the F1 measure was 0.474, and the maximum F1 measure achieved for the project (i.e., 7th iteration) was 0.557. The F1 measure achieved in the 3rd iteration represents 85\% of the maximum F1 measure achieved in the 7th iteration, and this percentage was reached using 1,499 comments (i.e., from 3 projects) whereas the maximum F1 measure used 2,404 comments (i.e., from 7 projects) in the training dataset. Therefore, for Ant we could achieve 85\% of the maximum result using only 62\% of the comments. We argue that the third iteration provides a good tradeoff between prediction performance and number of comments used to create the training dataset.

Because the results are differ for different projects, we also analyzed the iterations projects to determine the iterations that achieve the best performance across all projects. To measure that, we calculate the average percentage of the maximum F1 measure for each iteration. For example, we take the average percentage of the maximum F1 measure achieved during the 1st iteration for all projects, then we calculate the same value for all 2nd iterations and so on. We find that, the best performance is achieved during the 8th iteration, with an average maximum F1-measure of 96.57\% using (on average) 2,353 comments to create the training dataset. In comparison, the 9th iteration has an average  maximum F1-measure of 95.99\%, which is slightly lower than the average obtained in the 8th iteration, and uses more comments in the training dataset (i.e., 2,432). 
%As mentioned before, the addition of more comments not necessary implies more performance.

Table \ref{tbl:design_iteration_performance} shows the average percentage of the maximum F1-measure for each iteration. The first column shows the iteration number. The second column shows the average percentage of the maximum F1-measure achieved for each iteration. The third column presents the delta interval of the average percentage of the maximum F1-measure between one iteration and another. The fourth column shows the average of comments used in the training dataset of that specific iteration. Table \ref{tbl:design_iteration_performance} shows that, on average, we can achieve more than 90\% and 80\% of the maximum F1-measure in the second and third iterations, respectively. To achieve more than 90\% and 80\% of the maximum F1-measure, we require 1,444 (64.14\% of total comments required in 7th iteration) and 1,106 (49.13\% of total comments required in 7th iteration) comments, respectively.

\noindent \textbf{Results - requirement debt:} We find that although there is variation in the F1 measure value during the first 3 iterations they are not so preeminent as the variation found in design \SATD analysis. The F1 measure in requirement \SATD tend to be more constant through the iterations, and the first iteration has already a high percentage of the maximum F1 measured achieved for each project. This shows that the way the developers indicate requirement debt does not vary between different application domains as much as in design debt. This uniformity in requirement \SATD comments allows a good classification even with a small number of comments in the training dataset. We elaborate more on this point later in Section~\ref{sec:discussion}.

Figure \ref{fig:implementation_argo_result} \emad{check the figure ref...seems wrong} shows the F1-measure for different iterations in ArgoUML. The highest F1-measure of 0.65 \emad {use 2 decimal points throughout the paper} is achieved in the third iteration. From Figure~\ref{fig:implementation_argo_result}, we observe that we achieve the results better 90\% and 80\% of the highest F1-measure in iterations \todo{x} and \todo{y}, respectively. The reduction in comments is \todo{add result} for the 90\% and 80\% of the maximum F1-measure, respectively.

%ArgoUML presented small increases in F1-measure during the first three iterations reaching the best result at the 3rd iteration, 0.648. However, there was low variation in the F1 measure performance between iterations 4th (0.605) to 9th (0.595). In the 1st iteration (0.561) the classifier was trained with 110 requirement \SATD comments, which means 31\% of the comments that where used in the 9th iteration (0.595). A reduction of 69\% of the necessary training data to achieve almost the same result in terms of F1 measure in this case. 

Table \ref{tbl:requirement_iteration_performance} shows the average percentage of the maximum F1-measure for each iteration. Unlike the case of design debt, for requirement debt, the best F1-measure is achieved in the first iteration. This shows that using as few as 380 comments, we can effectively detect requirement \SATD.

%, the delta interval of the average maximum F1 measure between each iteration and the average number of comments used to create the training dataset. We also analyzed the average percentage of the maximum F1 measure between all iterations across the projects. We find that the 1st iteration was the one with the highest average percentage of the maximum F1 measure achieving a value of 87.3\% followed by the 7th iteration, with 83.7\% of the maximum F1 measure. The training dataset of the 1st iteration sized, on average, 380 \SATD comments. In comparison, an average of 654 requirement \SATD comments were used in the training dataset representing 97\% of all requirement \SATD comments that we classified during this study. This means that the 1st iteration, on average, performed better than the other iterations using only 55\% of the available training dataset at our disposal.

\begin{table}[!thb]
    \begin{center}
        \caption{Average maximum F1 measure for Design Debt  for all projects}
        \label{tbl:design_iteration_performance}
        \begin{tabular}{l| c c c}
        \toprule
        \thead{Iteration\\Number} & \thead{Average\%\\of maximum\\F1 measure} & \thead{$\Delta$\\between\\iterations} & \thead{Average\\comments} \\
        \midrule
         1  &  0.718 &  -      & 756   \\  
         2  &  0.856 & 0.138   & 1,106 \\  
         3  &  0.924 & 0.068   & 1,444 \\  
         4  &  0.912 & -0.012  & 1,717 \\  
         5  &  0.927 & 0.016   & 1,919 \\  
         6  &  0.930 & 0.002   & 2,108 \\  
         7  &  0.963 & 0.029   & 2,251 \\  
         8  &  0.965 & 0.006   & 2,353 \\  
         9  &  0.959 & -0.002  & 2,432 \\  
        \bottomrule
        \end{tabular}
    \end{center}    
\end{table}

\begin{table}[!thb]
	\begin{center}
		\caption{Average maximum F1 measure for Requirement Debt for all projects}
		\label{tbl:requirement_iteration_performance}
		\begin{tabular}{l| c c c}
			\toprule
			\thead{Iteration\\Number} & \thead{Average\%\\of maximum\\F1 measure} & \thead{$\Delta$\\between\\iterations} & \thead{Average\\comments} \\
			\midrule
			1  &  0.873 &   -      &  380 \\  
      2  &  0.772 & -0.101   &  481 \\
      3  &  0.778 & 0.006    &  541 \\  
      4  &  0.806 & 0.028    &  588 \\
      5  &  0.795 & -0.011   &  620  \\
      6  &  0.819 & 0.024    &  638  \\
			7  &  0.837 & 0.018    &  654  \\  
      8  &  0.833 & -0.004   &  668  \\  
      9  &  0.805 & -0.028   &  681  \\  
			\bottomrule
		\end{tabular}
	\end{center}    
\end{table}

%It is important to notice that, this results shows that it is possible to identify \SATD with a lower number of comments. Using a lower number of comments to create the training dataset makes the approach more applicable as the number of \SATD comments is rather scarce.    


The results of this RQ show that, contrary to our initial intuition, more data need not result in higher classification accuracy. In fact, we find that in some cases, the addition of more comments can decrease the performance of the classifier. One of the reasons for this is that the weight of features is given through empirical probability, and consequently features that appear more will have a higher weight. Although this is an effective process for the majority of cases that we studied, it can be misleading when classifying comments that have different contexts, i.e., in cross-project classification.


\conclusionbox{We find that using our NLP-based approach, design \SATD comments can be classified effectively using a training dataset of 1,106 - 1,444 comments. Similarly, requirement \SATD can be classified with as little as 380 comments of this type.}

\section{Discussion}
\label{sec:discussion}
\begin{figure*}[!thb]
  \centering
  \subfigure[Design Debt]{\includegraphics[width=0.48\textwidth]{figures/classifier_algorithms_comparison_design_1.pdf}
  \label{fig:algorithms_comparison_design}}
  \subfigure[Requirement Debt]{\includegraphics[width=0.48\textwidth]{figures/classifier_algorithms_comparison_implementation_1.pdf}
  \label{fig:algorithms_comparison_requirement}}
  \caption{Classification algorithms performance comparison}
  \label{fig:algorithms_comparison}
\end{figure*}

Thus far, we have seen that our NLP-based approach can perform well in classifying \SATD. However, there are some observations that warrant further investigation. For example, when it comes to the different types of \SATD, we find that requirement debt tends to require less training data, which is another interesting point that is worth further investigation. Lastly, when performing our classification, there are a number of underlying classifiers that can be used in the Stanford Classifier, hence we investigate what is the impact of using a different underlying classifier. 
%In this paper we propose an approach to identify \SATD comments using NLP. This tool, once trained correctly, can automatically classify natural language text. We create a training dataset of \SATD comments and analyzed the classification performance across ten open source projects. In RQ1, we show that our approach can outperform the current state-of-the-art in 10 out of 10 projects while identifying design and requirement debt. However, is not clear the reason why our approach was not equally effective across all projects. For example, JEdit has the worst F1 measure of all projects while classifying requirement \SATD.

%\subsection{Investigating Outliers Projects}
%By investigating JEdit comments, we notice two main reasons. First, the project has 10,322 comments. Of those only 14 are requirement \SATD. The data distribution represents a challenge to the classification. Even though the dataset is very unbalanced our precision was 0.125, which compared with the simple (random) baseline precision of 0.001, shows that our approach is still much more useful. Second, most requirement \SATD comments in this project are in the middle of long comments. 
%
%\everton {Since our approach creates a high number of prediction features for \SATD comments and also for comments without technical debt (i.e., positive and negative weight features), long comments have more chances to match a larger set of negative weight features that will lead to the without technical debt classification. Therefore, long comments can generate `noise' that hinders the classification performance of requirement \SATD. Based on our dataset, requirement \SATD comments are short comments having, on average, 81 characters. However, the average size of requirement \SATD comments from JEdit is 149 characters, 1.8$\times$ bigger than the overall average. Assuming that long requirement \SATD comments can be found in more projects that are not analyzed in our work, one can add these hypothetical comments to increment the current training dataset. Nevertheless, based on the projects that were analyzed, we argue that the current training dataset can identify requirement \SATD effectively and long requirement \SATD should be treated as exceptions.} 


% \nikos{I don't get this point}

% \emad{I do not see the point of the 2 paragraphs below}

% \emad{begin: remove} Intuitively we know that each project has its own particularities, and that each group of developers, must often, create a unique way to communicate their concerns with each other. This unique trait of source code comments is inherited from the natural language itself and renders the fully automated prediction of every single \SATD very unlikely. Even when analyzing a old aged project, changes in the context of the application and turnover of developers can reflect changes in the way that source code comments are written. We can notice the impact of these particularities on the detailed performance analysis conducted in RQ3, where we can notice that the addition of more comments can eventually decrease the F1 measure performance.

% For a great portion of \SATD comments there are common traits. Words as `workaround', `hack' are commonly imbued with criticism and the developers sense that this is not the appropriate solution for the problem in hand. However, relying just in these words for the identification of \SATD is not good enough as shown in Figures \ref{fig:f1_measure_comparison_design_debt} and \ref{fig:f1_measure_comparison_requirement_debt}. Therefore, NLP techniques, as proposed in our work, are needed in order to effectively identify \SATD comments.
% \emad{end: remove}



\subsection{Textual Similarity for Design and Requirement Debt}

% tf-idf formula \(tf-idf_{t,d}=tf_{t,d}\times idf_{t} \)
 % \cite{Manning2008book}

In RQ2, we hypothesize that one of the reasons that requirement \SATD comments need less training data is because requirement \SATD comments are more similar to each other than design \SATD comments. Therefore, we set out to measure the intra-similarity between the requirement debt comments and the intra-similarity of design debt comments and compare the two.

% The tf-idf weight is the resulting computation of two other metrics: \textit{term frequency} (tf) and \textit{inverse document frequency} (idf).

We start by calculating the term frequency-inverse document frequency \textit{tf-idf} weight of each design and requirement \SATD comment. The tf is the simple count of occurrences that a term (i.e., word) has in a document (i.e., comment). The idf takes into account the number of documents that the term appears. However, as the name implies, the more one term is repeated across multiple documents the less relevant it is. Therefore, let \textit{N} be the total number of documents in a collection, the inverse document frequency (idf) of a term \textit{t} is defined as follows: \(idf_{t} = log\frac{N}{df_{t}}\). The total tf-idf weight of a document is equals to the sum of each individual term tf-idf weight in the document. Each document is represented by a \textit{document vector} in a \textit{vector space model}. 

Once we have the tf-idf weights for the comments, we calculate the \textit{cosine similarity} between the comments. The Cosine similarity can be viewed as the \textit{dot product} of the normalized versions of two document (i.e., comment) vectors \cite{Manning2008book}. The computation of the cosine distance will vary between 0 to 1, where 0 means that the comments are not similar at all and 1 means that the comments are identical.

%To validate our hypothesis that requirement \SATD comments are more similar to each other than design \SATD comments, we created two separated datasets: one for each type of mentioned \SATD. Then, we apply to each comment the same preprocessing applied to comments in the training and test datasets as described in Subsection \ref{sub:run_the_nlp_classifier}. In addition, we removed English stop words from the comments as the NPL classification tool would do during the process of feature extraction. Then, we calculate the tf-idf weight and the cosine similarity for each comment. 

After calculating the tf-idf we compare each comment to all other comments of the same class and store its cosine similarity. For example, the requirement \SATD dataset contains 757 comments, which will generate a 757$\times$757 matrix (since we compare each comment to all the other comments). Fianlly, we took the average cosine similarity of each comment of the same type for design and requirement debt and plotted the distribution in Figure \ref{fig:textual_similarity}. Figure \ref{fig:textual_similarity} shows that the median and the upper quartile for requirement \SATD comments are higher than the median and upper quartile for design \SATD. The median for requirement debt comments is 0.018, whereas, the median for design debt comments is 0.011. To ensure that the difference is statistically significant, we perform a Wilcox test to calculate the p-value. The calculate p-value was \textless 2.2e-16 showing that our results is indeed statistically significant (i.e., p \textless 0.001). Considering our findings, our hypothesis is validated, showing that requirement \SATD comments are more similar between them than design \SATD comments are. This may help explain why requirement debt needs a smaller set of textual features to be detected.

\begin{figure}[t]
  \centering
  \includegraphics[width = 0.48\textwidth]{figures/textual_similarity_removing_stop_words.pdf}
  \vspace{-3mm}
  \caption{Textual Similarity Between Design and Requirement Debt Comments}
  \label{fig:textual_similarity}
\end{figure}

\subsection{Investigating the Impact of the Underlying Classifier of the NLP Classification}

In our work, all the classification done by the Stanford Classifier used a Logistic Regression classifier. However, the Stanford Classifier can use other classifiers. In order to examine the impact of the underlying classifier on our results, we chose the other two algorithms to execute the classification with, namely the Naive Bayes generative classifier and the Binary classifier.

Figures \ref{fig:algorithms_comparison_design} and \ref{fig:algorithms_comparison_requirement} compares the performance between the three different algorithms. We find that the Naive Bayes has the worst average F1-measure of 0.30 and 0.05 for design and requirement technical debt, respectively. Based on our findings, the Naive Bayes algorithm favors recall at the expense of precision. For example, while classifying design debt, the average recall was of 0.84 and precision 0.19. The two other algorithms present more balanced results compared to Naive Bayes, and the difference in performance between them is not as accentuate. The Logistic Regression classifier achieved F1-measures of 0.62 and 0.40, while the binary classifier F1-measures for design and requirement \SATD is 0.63 and 0.40, respectively. Tables \ref{tbl:improvement_f1measure_between_classifiers_design} and \ref{tbl:improvement_f1measure_between_classifiers_requirement} in the Appendix section provide detailed data for each compared algorithm.  

Although the Binary classification has a slightly better performance, for our research purpose, the Logistical Regression algorithm provide more insightful features as outcome. These features were analyzed and presented in RQ2. 


\section{Related Work}
\label{sec:related_work}
Our work uses code comments to detect \SATD through the use of a NLP technique. Therefore, we divide the related work into three categories: source code comments, technical debt and Natural Language Processing in Software Engineering.

\subsection{Source Code Comments}

A number of studies examined the co-evolution of source code comments and the rationale for changing code comments. For example, Fluri \textit{et al.}~\cite{Fluri2007WCRE} analyzed the co-evolution of source code and code comments, and found that 97\% of the comment changes are consistent. Tan \textit{et al.}~\cite{Tan2012ICST} proposed a novel approach to identify inconsistencies between Javadoc comments and method signatures. Malik \textit{et al.} \cite{Malik2008ICSM} studied the likelihood of a comment to be updated and found that call dependencies, control statements, the age of the function containing the comment, and the number of co-changed dependent functions are the most important factors to predict comment updates.

Other works used code comments to understand developer tasks. For example. Storey \textit{et al.}~\cite{Storey2008ICSE} analyzed how task annotations (e.g., TODO, FIXME) play a role in improving team articulation and communication. The work closest to ours is the work by Potdar and Shihab~\cite{Potdar2014ICSME}, where code comments were used to identify technical debt. 

Similar to some of the prior work, we also use source code comments to identify technical debt. However, our main focus is on the detection of different types of \SATD. As we have shown, our approach yields different and better results in detection of \SATD.
%Furthermore, we propose an approach to identify \SATD, that is derived from source code comments and natural language processing techniques, to detect \SATD.

\subsection{Technical Debt}

A number of studies have focused on detection and management of technical debt. For example, Seaman \textit{et al.}~\cite{Seaman2011}, Kruchten \textit{et al.}~\cite{Kruchten2013IWMTD} and Brown \textit{et al.}~\cite{Brown2010MTD} make several reflections about the term technical debt and how it has been used to communicate the issues that developers find in the code in a way that managers can understand. 

Other work focused on the detection of technical debt. Zazworka \textit{et al.} \cite{Zazworka2013CSE} conducted an experiment to compare the efficiency of automated tools in comparison with human elicitation regarding the detection of technical debt. They found that there is a small overlap between the two approaches, and thus it is better to combine them than replace one with the other. In addition, they concluded that automated tools are more efficient in finding defect debt, whereas developers can realize more abstract categories of technical debt.

In a follow up work, Zazworka \textit{et al.}~\cite{Zazworka2011MTD} conducted a study to measure the impact of technical debt on software quality. They focused on a particular kind of design debt, namely, God Classes. They found that God Classes are more likely to change, and therefore, have a higher impact on software quality. Fontana \textit{et al.}~\cite{Fontana2012MTD} investigated design technical debt appearing in the form of code smells. They used metrics to find three different code smells, namely God Classes, Data Classes and Duplicated Code. They proposed an approach to classify which one of the different code smells should be addressed first, based on its risk. Ernst \textit{et al.} ~\cite{Ernst2015FSE} conducted a survey with 1,831 participants and found that architectural decisions are the most important source of technical debt.

Our work is different from the work that uses code smells to detect design technical debt since we use code comments to detect technical debt. Moreover, our approach does not rely on code metrics and thresholds to identify technical debt and can be used to identify bad quality code symptoms other than bad smells.

More recently, Potdar and Shihab~\cite{Potdar2014ICSME} extracted the comments of four projects and analyzed more than 101,762 comments to come up with 62  patterns that indicates self-admitted technical debt. Their findings shows that 2.4\% - 31\% of the files in a project contain self-admitted technical debt. Our earlier work ~\cite{Maldonado2015MTD} examined more than 33 thousands comments to classify the different types of \SATD found in source code comments. Farias \textit{et al.} ~\cite{Farias2015MTD} proposed a contextualized vocabulary model for identifying TD on comments that uses word classes and code tags in the process. 

Our work also uses code comments to detect design technical debt. However, we use these code comments to train a NLP Classifier to automatically identify technical debt. Also, our focus is on \emph{self-admitted} design and requirement technical debt.

\subsection{NLP in Software Engineering}

A number of studies leveraged NLP in software engineering, mainly for the traceability of requirements, program comprehension and software maintenance. For example, Lormans and van Deursen~\cite{Lormans2006CSRM} used latent semantic indexing (LSI) to create traceable links between requirements and test cases and requirements to design implementations. Hayes \textit{et al.}~\cite{Hayes2005, Hayes2006TSE} created a tool called RETRO that applies information retrieval techniques to trace and map requirements to designs. Yadla \textit{et al.}~\cite{yadla2005tracing} further enhanced the RETRO tool and linked requirements to issue reports. On the other hand, Runeson \textit{et al.}~\cite{Runeson2007ICSE} implemented a NLP-based tool to automatically identify duplicated issues reports, they found that 2/3 of the possible duplicates examined in their study can be found with their tool. Canfora and Cerulo \cite{Canfora2005ISSM} link a change request with the correspondent set of source files using NLP techniques. The performance of the approach is evaluated on four open source projects.  

The prior work motivated us to use NLP techniques. However, our work is different from the aforementioned since we apply NLP techniques on code comments to identify \SATD, rather than use it for traceability.

\section{Conclusion and Future work}
\label{sec:conclusion}
Technical debt is a term being used to express none optimal solutions, such as hacks and workarounds, that are applied during the software development process. Although these none optimal solutions can help achieve immediate pressing goals, most often they will have a negative impact on the project maintainability~\cite{Zazworka2011MTD}. 

Our work focuses on the identification of \SATD through the use of Natural Language Processing. We analyzed the comments of 10 open source projects namely Ant, ArgoUML, Columba, EMF, Hibernate, JEdit, JFreeChart, Jmeter, JRuby and SQuirrel SQL. These projects are considered well commented and they belong to different application domains. The comments of these projects were manually classified into specific types of technical debt such as design, requirement, defect, documentation and test debt. Next, we selected 61,664 comments from this dataset (i.e., those classified as design \SATD, requirement \SATD and without technical debt) to train the NLP Classifier, and then this tool was used to identify  design and requirement \SATD automatically.
% \nikos{I think we didn't give the entire dataset as input, but only the design/requirement debt comments and no-debt comments}

We first evaluated the performance of our approach by comparing our F1 measure with two other baseline F1 measures, i.e., the comment patterns baseline and the simple (random) baseline. We shown that our approach outperforms the state-of-the-art on average 4.5 times in the identification of design \SATD. Moreover, our approach can identify requirement \SATD, deed that the state-of-the-art fails to achieve. Our approach performance surpasses the simple (random) baseline on average 7.1 and 18 times for design and requirement \SATD, respectively. Then, we explored the characteristics of the features (i.e., words) used to classify \SATD. We find that the words used to express design and requirement \SATD are different from each other. The three strongest indicators of design \SATD are `hack', `workaround' and `yuck!', whereas, `todo', `needed' and `implementation' are the strongest indicators of requirement debt. In addition, we find that even using a low number of \SATD comments in the training dataset can achieve high classification performance. In fact, our results show that developers use a richer vocabulary to express design \SATD and a training dataset of at least 1,444 design \SATD comments is necessary to obtain a satisfactory classification. On the other hand, requirement \SATD is expressed in a more uniform way, and with a training dataset of 380 \SATD comments it is possible to classify with success requirement \SATD automatically.

In the future, we believe that more analysis is needed to fine tune the use of the current training dataset in order to achieve maximum efficiency in the prediction of \SATD comments. For example, using subsets of our training dataset can be more suitable for some applications than using the whole dataset due to domain particularities. However, the results thus far are not to be neglected as our approach has the best F1-measure performance on every analyzed project. Moreover, to enable future research, we make the dataset created in this study publicly available. We believe that it will be a good starting point for researchers interested in identifying technical debt through comments and even using different Natural Language Processing techniques on it. Lastly, we plan to use the findings of this study to build a tool that will support software engineers in the task of identifying and managing \SATD. 


% \emad{we have 3 results there....we really need to simply this and have 1 or max 2 results.}

\bibliographystyle{IEEEtran}
\balance
\bibliography{design_td}

\appendix{}
\label{sec:appendix}

Table~\ref{tab:comment_patterns} lists all of the comment patterns derived in our study. If the paper is accepted, this appendix will be provided through an online link. We include this appendix, which will be removed later, so that reviewers need not go to an online link during the review process. Online links have been discouraged in the past during the review process since they may indicate the identity of the anonymous reviewers.


\begin{table*}[thb!]
	\begin{center}
		\caption{The Derived 176 Comment Patterns that Indicate \SADTD}
		\label{tab:comment_patterns}
		\begin{tabular}{l| l| l| l }
			\toprule
			`\%future\%may\%'            & `\%perhaps\%elsewhere\%'    & `\%hard\%coding\%'            & `\%todo\%complex\%'       \\
			`\%future\%better\%'         & `\%rather\%complex\%'       & `\%kludge\%'                  & `\%fixme\%complex\%'      \\
			`\%future\%enhance\%'        & `\%held\%?\%'               & `\%todo\%public\%'            & `\%xxx\%complex\%'        \\
			`\%future\%change\%'         & `\%though\%unused\%'        & `\%fixme\%public\%'           & `\%consistency\%sake\%'   \\
			`\%quick\%fix\%'             & `\%todo\%don\%know\%'       & `\%xxx\%public\%'             & `\% lack \%broke\%'       \\
			`\%temporary\%until\%'       & `\%fixme\%don\%know\%'      & `\%messy\%'                   & `\% lack \%problem\%'     \\
			`\%place\%somewhere\%else\%' & `\%xxx\%don\%know\%'        & `\%should\%instead\%'         & `\% lack \%should\%'      \\
			`\%move\%somewhere\%else\%'  & `\%don\%know\%try\%'        & `\%this\%weird\%'             & `\%todo\% lack \%'        \\
			`\%used\%other\%place\%'     & `\%don\%know\%fail\%'       & `\%weird\%this\%'             & `\%fixme\% lack \%'       \\
			`\%it \%may \%change \%'     & `\%don\%know\%what\%'       & `\%todo\%weird\%'             & `\%xxx\% lack \%'         \\
			`\%this may change\%'        & `\%don\%know\%fix\%'        & `\%fixme\%weird\%'            & `\%todo\% long \%'        \\
			`\%todo\%can\%change\%'      & `\%not\%fond\%'             & `\%xxx\%weird\%'              & `\%fixme\% long \%'       \\
			`\%fixme\%can\%change\%'     & `\%more\%elegant\%'         & `\%todo\%availability\%'      & `\%xxx\% long \%'         \\
			`\%xxx\%can\%change\%'       & `\%clean\%way\%'            & `\%fixme\%availability\%'     & `\%todo\% large \%'       \\
			`\%not \%sure \%'            & `\%todo\%remove\%'          & `\%xxx\%availability\%'       & `\%fixme\% large \%'      \\
			`\%dependency\%cycle\%'      & `\%xxx\%remove\%'           & `\%todo\%extensibility\%'     & `\%xxx\% large \%'        \\
			`\%todo\%dependenc\%'        & `\%fixme\%remove\%'         & `\%fixme\%extensibility\%'    & `\%future\%maintenance\%' \\
			`\%fixme\%dependenc\%'       & `\%todo\%don\%want\%'       & `\%xxx\%extensibility\%'      & `\%todo\%maintenance\%'   \\
			`\%xxx\%dependenc\%'         & `\%fixme\%don\%want\%'      & `\%sacrifice\%flexibility\%'  & `\%fixme\%maintenance\%'  \\
			`\%code\%cop\%from\%'        & `\%xxx\%don\%want\%'        & `\%todo\%flexibility\%'       & `\%xxx\%maintenance\%'    \\
			`\%copied\%code\%'           & `\% fix \% for \%'          & `\%fixme\%flexibility\%'      & `\%todo\%unused\%'        \\
			`\% any \%reason\%'          & `\% fix for \%'             & `\%xxx\%flexibility\%'        & `\%fixme\%unused\%'       \\
			`\%wrong\%place\%'           & `\%irritating\%'            & `\%todo\%scalability\%'       & `\%xxx\%unused\%'         \\
			`\%hairy\%'                  & `\%todo\%duplicat\%'        & `\%fixme\%scalability\%'      & `\%currently\%unused\%'   \\
			`\%instead\%could\%'         & `\%fixme\%duplicat'         & `\%xxx\%scalability\%'        & `\%unused\%delete\%'      \\
			`\%ugly\%'                   & `\%xxx\%duplicat\%'         & `\%security\%compatibility\%' & `\%unused\%currently\%'   \\
			`\%todo\%avoid\%'            & `\%why\%not\%'              & `\%security\%never\%'         & `\%such\%bad\%'           \\
			`\%fixme\%avoid\%'           & `\%rethink\%'               & `\%todo\%security\%'          & `\%todo\%bad\%'           \\
			`\%xxx\%avoid\%'             & `\%rework\%'                & `\%fixme\%security\%'         & `\%fixme\%bad\%'          \\
			`\%should\%avoid\%'          & `\%pointless\%'             & `\%xxx\%security\%'           & `\%xxx\%bad\%'            \\
			`\%pathological\%'           & `\% not \%nice\%'           & `\%todo\%ambiguous\%'         & `\%todo\%clone\%code\%'   \\
			`\%stolen\%'                 & `\%hack\%'                  & `\%fixme\%ambiguous\%'        & `\%fixme\%clone\%code\%'  \\
			`\%not\%well\%formed\%'      & `\%only\%developer\%know\%' & `\%xxx\%ambiguous\%'          & `\%xxx\%clone\%code\%'    \\
			`\% no \%sense\%since\%'     & `\% use \% help\%'          & `\%todo\% big \%'             & `\% dead \%code\%'        \\
			`\%without\%notic\%'         & `\%hammer\%'                & `\%fixme\% big \%'            & `\%crappy\%design\%'      \\
			`\%brittle\%'                & `\%todo\%redundant\%'       & `\%xxx\% big \%'              & `\%design\%flaw\%'        \\
			`\%really\%necessary\%'      & `\%fixme\%redundant\%'      & `\% big \% mess \%'           & `\% todo\% design \%'     \\
			`\%cares\%'                  & `\%xxx\%redundant\%'        & `\%clean\%needed\%'           & `\%fixme\%design\%'       \\
			`\%no idea\%'                & `\%for\%some\%reason\%'     & `\%should\%clean\%'           & `\% xxx \%design\%'       \\
			`\%idea?\%'                  & `\%alternatively\%could\%'  & `\%todo\%clean\%'             & `\%redesign\%'            \\
			`\%doing?\%'                 & `\%technically\%'           & `\%fixme\%clean\%'            & `\%todo\%magic\%'         \\
			`\%todo\%elsewhere\%'        & `\% forces \%us\%'          & `\%xxx\%clean\%'              & `\%fixme\%magic\%'        \\
			`\%fixme\%elsewhere\%'       & `\%better\%way\%'           & `\%due\%complex\%'            & `\%xxx\%magic\%'          \\
			`\%xxx\%elsewhere\%'         & `\%hard\%coded\%'           & `\%way\%complex\%'            & `\%smell\%'               \\ 
			\bottomrule
		\end{tabular}
	\end{center}
\end{table*}

\end{document}
