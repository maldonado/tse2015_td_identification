


Thus far, our study has focused on determining comment patterns that can be used to detect \SADTD and examining the effectiveness of our comment patterns in detecting \SADTD. However, how our approach compares to existing approaches is an interesting question that is left unanswered. Therefore, in this section, we compare our approach to existing state-of-the-art approaches that have been used to detect design technical debt~\cite{Zazworka2011MTD}. Then, we discuss the applicability of using refactoring recommendation systems to address \SADTD. 


%\subsection{Applicability of \SADTD~Comment Patterns on Different Projects}
%
%When determining the comment patterns to detect \SADTD, we analyzed the comments of the Apache Ant project. As shown in Section~\ref{sec:case_study_results}, the precision and recall values for the Apache Ant project is high. However, when the same comment patterns are applied to different projects, we are able to obtain high precision values (between 74.04 - 88.57\%), but the recall decreases significantly (between 10.87-27.16\%).
%
%This observations highlights an important issue with using our approach. Since software projects tend to use specific terms when describing their \SADTD, it is best to derive comment patterns from the same project. However, it is not all bad news. What our results show is that the comment patterns extracted from one project can achieve high precision. This means that the comment patterns from one project can be applied to another project, however, these comment patterns are likely to be conservative (i.e., achieving high precision), but miss many of the \SADTD~in the new/different project (i.e., the low recall).  One can improve recall by aggregating comment patterns from multiple projects, however, such an approach will impact the precision values.
%
%Constructing a global set of comment patterns that can be used to detect \SADTD~in all projects and examining the tradeoff between precision and recall is an area for future work, however, we see this work as a contribution in the right direction.
%



\subsection{Comparing our Comment-based Approach to the State-of-the-art in Design Technical Debt Detection}

Prior work by Zazworka \emph{et al.}~\cite{Zazworka2011MTD} was one of the first to focus on the detection of design technical debt. In their work, the authors use Marinescu's~\cite{Marinescu2004ICSM} detection strategies to identify God classes. It is assumed that God classes are strong indicators of design technical debt. Our approach aims to solve the same problem, i.e., the detection of design technical debt, however, we use code comments to answer detect the design technical debt. Therefore, one question that arises is: \emph{are we finding the same design technical debt or do the two approaches complement each other?} 

To answer this question, we implemented Marinescu's metrics in the JDeodorant tool to detect God classes in all of the analyzed projects. Then, we measure the overlap between the files that our approach flags as having \SADTD~and the files that contain God classes. A high overlap means that we are indeed finding the same design technical debt issues as the state-of-the-art work. A low overlap means that our approach complements the state-of-the-art approach, i.e., each approach finds different types of design technical debt.

Design technical debt has been studied before from several perspectives but, until now, not from the developers' comments perspective. Therefore, we would like to compare how our proposed approach compares with proved techniques used in previous studies \cite{Zazworka2011MTD}. Zazworka used Marinescu's \cite{Marinescu2004ICSM} detection strategies to identify god classes as they are a strong indicator to design technical debt. Thus, we used the same detection strategies to identify the god classes in each one of our projects and then we  compare the overlap with design technical debt comments found on these files.

Table~\ref{tab:godclasscomparison} shows that there is very little overlap between the design technical debt issues flagged by our approach and the prior approach using God classes. We observe that in the best case, which is for the Jmeter project, only 5 files overlap. Another important observation form Table~\ref{tab:godclasscomparison} is that in all projects, except for JFreeChart, our approach flags more files as having design technical debt. Based on this finding, we suggest that both approaches, i.e., our approach and the approach based on God classes, should be used to maximize the detection of technical debt.

\subsection{Using Refactoring Recommendation Systems to Mitigate \SADTD}
Thus far, all of our work has focused solely on the \emph{detection} of design technical debt. One question that still lingers is \emph{What can we do to mitigate this design technical debt?}  Many approaches can be employed to mitigate design technical debt, however, in this subsection, we focus on the use of refactoring recommendation systems to mitigate \SADTD.

In order to examine the applicability of using refactoring to mitigate \SADTD, we use the refactoring recommendations provided by JDeodorant. In particular, we run JDeodorant on all the projects. JDeodorant analyzed the source code and suggested four types of  refactoring opportunities: Extract Class, Type-check Elimination, Extract Method and Move Method. We passed in the class names identified using our approach as being \SADTD~and recorded whether JDeodorant would suggest a refactoring. Then, we check whether the refactoring opportunities suggested by JDeodorant are in the same method (using the method signatures) that we found the \SADTD~comment in. 

We find that out of 964 \SADTD, JDeodorant recommended refactoring opportunities for 237 methods that contain \SADTD~comments. The refactoring opportunities were distributed as follows: The tool identified fifteen Move Method opportunities. The Move Method refactoring can improve cohesion and reduce coupling by grouping methods that should be placed together. Ninety Extract Method opportunities were suggested. Extract Method refactoring improves code readability and increases the usability of the methods by breaking long methods into smaller methods. On the other hand, 132 methods with \SADTD were participating in an Extract Class refactoring opportunity. The Extract Class refactoring is applied when addressing the Single Responsibility Principle violations.

%\everton{Add a full-static example}

In total, 24.58\% of the methods containing \SADTD~comments had at least one refactoring opportunity suggested by the tool during our analysis.  


\begin{table*}[!hbt]
	\begin{center}
		\caption{God classes Vs. Design Technical Debt Comments}
		\vspace{-2mm}
		\label{tab:godclasscomparison}
		\begin{tabular}{l| c c c c} 
			\toprule
			\textbf{Project} & \textbf{Total classes} & \textbf{God classes} &  \textbf{Design TD Classes}  &  \textbf{Overlap}  \\ 
			\midrule
			Apache Ant & 1,475 & 11 & 63 & 1  \\ 
			Jakarta Jmeter & 1,181 & 19 & 60 & 5  \\
			ArgoUML & 2,609 & 25 & 193 & 3  \\
			Columba & 1,711 & 18 & 23 & 1  \\
			EMF & 1,458 & 25 & 21 & 0 \\
			Hibernate  & 1,356 & 13 & 78 & 0 \\
			JEdit & 800 & 7 & 55  & 0 \\
			JFreeChart & 1,065 & 9 & 9 & 0  \\
			JRuby & 1,486 & 23 & 76 & 0  \\
			SQuirrel  & 3,108 & 30 & 75 & 1 \\  
			\bottomrule
		\end{tabular}
	\end{center}
\end{table*}

%\noindent \textbf{Motivation:} Software engineers practitioners dedicate a large amount of time, effort and money in order to maintain their source code in a good working condition. As well the research community invest a loot \emad{loot?} of resources in this subject, as a result different tools were developed to support code maintenance and more specifically tools that helps the software engineer to analyze and improve the design of existent code. We want to analyze the overlap between self-admitted design technical debt and automated refactoring opportunities that our approach can reveal. That could show us that refactoring opportunities tools can take advantage of this light weight approach to detect design flaws in combination of the other analysis already implemented. 

%\emad{I would say that we should say 1. We know that design technical debt exists, 2. so how can we deal with it?, 3. one way is to use automated approaches, such as refactoring tools. 4. Now we examine how well these refactoring tools can help us deal with this design debt.}


%\noindent \textbf{Approach:} In the a previous section we shown how to use JDeodorant to extract the comments for our analysis, now we will see in detail how to use it to check for refactoring opportunities. \emad{Here, I would just say we use JDeodorant to help us determine these automated refactorings}

%JDeodorant can analyze the source code and suggest refactoring opportunities within these categories:Extracting Class refactoring opportunities, Type Check Elimination refactoring opportunities, Extract Method refactoring opportunities and Move Method refactoring opportunities \emad{maybe we should add a brief description of each of these in a table}. The tool also provides an interface that enables our program to call JDeodorant and check for the mentioned refactorings opportunities in the desired  fragment of code. This interface can be found in JDeodorant source code in the Standalone class. 

%We took advantage of this functionality to implement in our extraction tool one feature that analyses the comments that where identified as design technical debt. The Standalone mode of JDeodorant expects one "ClassObject" object as parameter in order to search for each one of the four available refactorings. So the first step for us is to associate the "CommentClass" of our extraction tool with the "ClassObject" expected by JDeodorant. Then we check if the refactorings opportunities suggested by the tool is in the same method that our design technical debt comment was found and store this result in the database \emad{we should discuss this in the threats section,i.e., our accuracy here is at the granularity of the method}.

%To identify if the method found in the refactoring opportunities is the same as the one that we found in our extraction tool we compare them using the method signature. We did small modifications in our version of JDeodorant to achieve that. We added methods to support this comparison for Extract Method refactoring candidates and for Extract Class refactoring candidates. The only exception is for Type Checking refactoring candidates that we check the name of the method instead of the method signature.   

%\par \noindent \textbf{Results:} In total our approach could identify 964 comments as containing design technical debt, JDeodorant found refactoring opportunities for 24.59\% of the comments, which means 237 comments. \emad{Please elaborate on this and say the type of refactoring and just elaborate on the results in general, e.g., what type of design debt can be automatically refactored, etc.}

%\emad{The idea here is to say that apron. 24.6\% of the self-admitted technical debt can be addressed by automatic refactorings.}

%\conclusionbox{Our findings shows that JDeodorant found refactoring candidates for 24.59\% of the methods classified as design technical debt by our approach.}




