The term Technical Debt is often used to express some kind of inadequacy in the source code in a way that is understandable to management. But this metaphor can represent many different kind of things, inappropriate or temporary solution to meet a deadline, error prone code, lack of tests and documentation and even design flaws or workarounds. Sometime, developers are aware of these problems and they may express their concern through comments in the source code. Therefore, in this study we propose an approach to identify such comments in the source code. Our findings show that:

\begin{itemize}
\item The derived 176 comment patterns can be used to effectively identify \SADTD, with precision values ranging between 74.07-96.30\% and recall values ranging between 10.87-83.87\%.
\item The design technical debt identified by out approach is different than the design technical debt detected through code smells. 

\item Approximately 24.58\% of the detected \SADTD can be automatically refactored with automated refactoring tools.

\end{itemize} 

%During this study we also presented 3 Heuristics and one post-processing technique to eliminate irrelevant comments, the development of a taxonomy and a 176 \SADTD~patterns to address design technical debt and finally we shown the effectiveness of refactoring tools dealing with design technical debt. 

We believe that our study lays the ground work for future work in the area of self-admitted technical debt. In the future, we plan to explore Natural Language Processing techniques in order to improve the detection of \SADTD across projects. Furthermore we plan to expand the current taxonomy to include more categories of technical debt like defect and test technical debt. Finally, we plan to explore the use of our technique to possibly rank refactoring candidates suggestions from automated tools based on the type of technical debt. 