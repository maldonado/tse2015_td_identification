Technical debt is a term being used to express non optimal solutions, such as hacks and workarounds, that are applied during the software development process. Although these non optimal solutions can provide a way to achieve immediate pressing goals, most often they will have a negative impact in the project maintainability, and will require an increased effort to be properly addressed in the long run. 

Our work focuses in the identification of \SATD through the use of Natural Processing Language. \SATD is the technical debt deliberately introduced by the developers and reported through source code comments.

We analyzed the comments of 10 open source projects namely Apache Ant, Apache Jmeter, ArgoUML, Columba, EMF, Hibernate, JEdit, JFreeChart, JRuby and SQuirrel. These projects are considered well commented and they belong to different application domains.

The comments of these projects where manually classified into specif types of technical debt such as design, requirement, defect, documentation and test debt. This dataset with more than 63,000 of classified comments were used to create training datasets to a maximum entropy classifier tool and then used to identify  design and requirement \SATD.

We find that our approach is 2 to 20 times more efficient in the identification of \SATD than the state-of-the-art approach. We achieved an average F1 measure of 0.62 and 0.51 while identifying design and requirement \SATD comments respectively. 

We analyzed strong indicators of \SATD in code comments and we ranked the top 10 words used by developers to express design and requirement debt in the studied projects. For design debt the most common indicators were `hack', `workaround', `kludge', `yuck!', `fixme', `todo', `stupidity', `ugly', `unused?' and `sucks'. Whereas, for requirement debt the most common words were: `todo', `needed', `implementation', `fixme', `xxx', `auto-generated', `ends?', `configurable', `convention' and `apparently'.
 
Regarding the amount of data necessary to effectively identify \SATD comments using our approach we find that training datasets with at least 1,444 design debt comments can be used to identify \SATD. Similarly, datasets with at least 1,055 requirement debt comments can be used for the purpose of requirement \SATD identification. 

In this work, we contribute with the dataset created in this study making it publicly available, we believe that it will be a good starting point for researchers interested in identifying technical debt trough comments and even using different types of Natural Processing Language techniques. 

In a future work we will use the findings of this study to build a tool that will support software engineers in the task of identifying and managing the \SATD portfolio. We believe that our dataset can be used towards the effective identification of \SATD comments and it will be beneficial to the developer to see technical debt that was pointed out by other humans instead of relying in metrics and thresholds. 