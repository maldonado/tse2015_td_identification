Developers often have to deal with conflicting goals that require software to be delivered quickly, with high quality, and on budget. In practice, achieving all of these goals at the same time can be challenging, causing a tradeoff to be made. Often, these tradeoffs lead developers to take \emph{shortcuts} or use \emph{workarounds}. Although such shortcuts help developers in meeting their short-term goals, they may have a negative impact in the long-term.

Technical debt is a metaphor that has been used to express sub-optimal solutions that are taken consciously in a software project in order to achieve some short-term goals. Generally, these decisions allow the project to move faster in the short-term, but introduce an increased cost (i.e., debt) to maintain this software in the long run~\cite{Seaman2011,Kruchten2013IWMTD}. Prior work showed that technical debt is widespread in the software domain, is unavoidable, and can have a negative impact on the quality of the software~\cite{Lim2012Software}.

Due to the importance of technical debt, a number of studies empirically examined it and proposed techniques to enable its detection and management. The main findings of the prior work is that 1) there are different types of technical debt, e.g., defect debt, design debt, testing debt, and that design debt has the highest impact~\cite{Alves2014MTD,Marinescu2012IBM}; and 2) statically analyzing the source code can help detecting technical debt~\cite{Marinescu2004ICSM,Marinescu2010CSMR,Zazworka2013CSE}. In particular, these works use metric thresholds to detect code smells, which are considered as proxies for technical debt. 

One major drawback of using metrics to detect technical debt is that no one knows if the detected smells really constitute technical debt, or if they correspond to problems that the developers care about. Therefore, more recently, our work showed that using code comments can be effective in identifying self-admitted technical debt~\cite{Potdar2014ICSME}. This work uses comments to detect \emph{generic} technical debt, and did not focus on any specific type of technical debt.